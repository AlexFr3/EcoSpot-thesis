\chapter*{Conclusioni}
\addcontentsline{toc}{chapter}{Conclusioni}
\markboth{Conclusioni}{Conclusioni}

Il presente lavoro di tesi ha portato alla progettazione e allo sviluppo di \textit{EcoSpot}, un’applicazione mobile cross-platform realizzata in Flutter, che si pone come strumento 
innovativo per la valorizzazione del Parco del Delta del Po.
L'obiettivo primario è stato raggiunto integrando tecnologie eterogenee  dalla cartografia digitale ai sensori IoT, fino ai servizi Cloud  in un'unica interfaccia accessibile, capace 
di promuovere il turismo sostenibile e la consapevolezza ecologica.

L'analisi dei risultati ottenuti dimostra come l'adozione di un'architettura a microservizi basata su Riverpod per la gestione dello stato e su Firebase come backend abbia 
garantito la scalabilità e la robustezza necessarie per gestire flussi di dati in tempo reale.
In particolare, l'implementazione del paradigma del \textit{Digital Twin}, che unisce i dati ambientali (livelli idrometrici, salinità) all'esperienza utente, ha permesso di rendere 
visibili le dinamiche invisibili dell'ecosistema lagunare, trasformando il visitatore in un soggetto attivo tramite la \textit{Citizen Science} 
\cite{2025discover}.

Dal punto di vista della \textit{User Experience}, il sistema di \textit{Gamification}, basato sulla \textit{Self-Determination Theory} \cite{ryan_deci_sdt}, ha gettato le basi per un 
coinvolgimento profondo dell'utente, incentivando l'esplorazione fisica dei luoghi tramite meccaniche di scoperta \textit{location-based}.
Tuttavia, EcoSpot è un progetto in continua evoluzione; l'architettura modulare adottata permette di delineare diverse direttrici per sviluppi futuri volti ad arricchire ulteriormente 
l'esperienza.

\section*{Sviluppi Futuri}

In ottica evolutiva, le funzionalità attuali possono essere estese introducendo nuove dinamiche di gioco e strumenti di assistenza alla navigazione, per massimizzare la \textit{retention} dell'utente e l'impatto educativo.

\subsection*{Sistema di Missioni e Badge Avanzati}
Attualmente il sistema gratifica la scoperta singola con Punti Esperienza (XP) e l'avanzamento di Rango. Un'evoluzione naturale prevede l'introduzione di \textbf{Missioni Giornaliere e Stagionali} (es. "Fotografa 3 specie di uccelli acquatici in una settimana" o "Visita tutte le stazioni idrometriche delle Valli di Comacchio").

A supporto di ciò, il sistema di Badge (attualmente legato al rango) potrà essere espanso con riconoscimenti specifici per categorie, come:
\begin{itemize}
    \item \textbf{Il Guardiano delle Acque}: per chi consulta frequentemente i dati dei sensori IoT.
    \item \textbf{Bio-Reporter}: per gli utenti che inviano il maggior numero di segnalazioni validate.
    \item \textbf{Esploratore dell'Alba}: per chi visita i punti di osservazione negli orari migliori per l'avifauna.
\end{itemize}

\subsection*{Percorsi Personalizzati e Routing Intelligente}
Per migliorare la fruibilità del territorio, si prevede l'integrazione di un sistema di \textit{routing} intelligente. A differenza della libera esplorazione attuale, l'app potrà generare itinerari \textit{turn-by-turn} personalizzati in base agli interessi dell'utente (es. "Percorso fenicotteri" o "Itinerario storico delle saline") e alle sue capacità motorie (es. percorsi accessibili in bicicletta o sedia a rotelle).

L'algoritmo potrà suggerire il percorso ottimale incrociando i dati statici della mappa con i dati dinamici dei sensori e delle segnalazioni recenti: se un utente ha appena avvistato una specie rara in una zona, il sistema potrà proporre quel punto come meta suggerita ad altri utenti nelle vicinanze.

\subsection*{Riconoscimento Automatico tramite AI}
Per potenziare il lato \textit{Citizen Science} e ridurre le barriere all'ingresso per i non esperti, si ipotizza l'integrazione di modelli di \textit{Machine Learning} (es. TensorFlow Lite) direttamente \textit{on-device}. Questa funzionalità permetterebbe il riconoscimento automatico o assistito della specie inquadrata dalla fotocamera durante la fase di segnalazione, validando istantaneamente la scoperta e riducendo il carico di moderazione lato server.

\subsection*{Funzionalità Social e Leaderboard}
Facendo leva sul bisogno di "Relazione" descritto nella teoria SDT, l'app potrà implementare classifiche (\textit{Leaderboard}) globali o tra amici, basate sui punti XP accumulati. La condivisione delle proprie "Carte Collezione" (le specie sbloccate) sui social network esterni fungerà inoltre da volano promozionale per il Parco stesso.

\subsection*{Realtà Aumentata (AR)}
Infine, sfruttando i sensori del dispositivo mobile già in uso (GPS, bussola), si potrà sviluppare una modalità in Realtà Aumentata. Inquadrando il paesaggio, l'utente vedrà sovrapposti i dati dei sensori IoT (es. il livello dell'acqua visualizzato direttamente sopra il canale) o le sagome virtuali della fauna presente in quell'habitat, rendendo l'esperienza immersiva e rafforzando il concetto di gemello digitale del territorio.