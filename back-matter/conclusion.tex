\chapter*{Conclusioni}
\addcontentsline{toc}{chapter}{Conclusioni}
\markboth{Conclusioni}{Conclusioni}

Il presente lavoro di tesi ha portato alla progettazione e allo sviluppo di \textit{EcoSpot}, un’applicazione mobile cross-platform realizzata in Flutter che si propone come uno 
strumento innovativo per la valorizzazione del Parco del Delta del Po. L'obiettivo primario, ovvero, di promuovere attivamente il turismo sostenibile e la consapevolezza ecologica 
è stato raggiunto integrando tecnologie eterogenee dalla mappa ai sensori IoT, fino ai servizi Cloud in un'unica interfaccia intuitiva ed accessibile.

In sintesi, l'analisi dell'applicativo sviluppato dimostra il pieno raggiungimento degli obiettivi prefissati in fase di progettazione, 
i quali possono essere riassunti in quattro macrocategorie fondamentali. In primo luogo, i requisiti di architettura e gestione dati sono stati rispettati implementando 
un'infrastruttura Cloud robusta, in grado di gestire in sicurezza l'autenticazione degli utenti, la persistenza delle scoperte e lo storage dei contenuti multimediali.
In secondo luogo, per quanto concerne l'integrazione di servizi esterni e IoT, l'applicazione comunica efficacemente con i servizi di geolocalizzazione, con le API meteorologiche 
e con la rete di sensori ambientali distribuiti sul territorio, garantendo una visualizzazione affidabile e in tempo reale.
Sotto il profilo dell'interfaccia e usabilità, l'adozione di una mappa interattiva dinamica e di schede informative modulari ha permesso di offrire un'esperienza fluida, 
intuitiva e accessibile. Infine, i requisiti di coinvolgimento e cittadinanza attiva sono stati ampiamente soddisfatti attraverso l'introduzione di meccanismi di gamification 
basati su punti esperienza e livelli, uniti alla possibilità di inviare segnalazioni georeferenziate. 
L'obiettivo finale di queste scelte è proprio quello di coinvolgere il turista, facendolo sentire parte integrante del territorio, 
in modo da incentivare un comportamento responsabile verso il parco e aiutare concretamente la sua salvaguardia. Così facendo, viene 
pienamente soddisfatto il paradigma della Citizen Science, trasformando l'utente in una vera e propria risorsa attiva per il monitoraggio e la tutela dell'ecosistema.

Se da un lato l'integrazione di queste meccaniche di scoperta \textit{location-based} fondate sui principi della \textit{Self-Determination Theory} 
ha gettato solide basi per un coinvolgimento profondo del visitatore, dall'altro EcoSpot si configura come un punto di partenza. 
L'architettura modulare adottata nel progetto offre, infatti, varie opportunità per sviluppi futuri volti ad arricchire ulteriormente 
l'esperienza immersiva all'interno del Parco.

\section*{Sviluppi Futuri}

In ottica evolutiva, le funzionalità attuali possono essere estese introducendo nuove dinamiche di partecipazione per massimizzare la fidelizzazione dell'utente e l'impatto educativo 
del sistema. Mentre il prototipo attuale gratifica la scoperta singola con l'avanzamento di rango, un'evoluzione naturale prevede l'introduzione di missioni giornaliere e stagionali. 
Queste sfide, come la richiesta di documentare specifiche specie in un arco temporale definito o visitare determinate aree verdi, potrebbero essere accompagnate da un meccanismo di 
medaglie di completamento. L'obiettivo è quello di incentivare la scoperta dell'intera biodiversità mappata, trasformando la visita in una sfida coinvolgente per completare il proprio 
archivio digitale e favorendo un coinvolgimento costante nel lungo periodo.

Per migliorare la fruibilità del territorio, si prevede inoltre l'integrazione di un sistema di \textit{routing} intelligente. A differenza della libera esplorazione, l'app potrebbe 
generare itinerari personalizzati basati sugli interessi specifici dell'utente o sulle sue capacità motorie, suggerendo percorsi accessibili e ottimizzati. Tale algoritmo potrebbe 
persino diventare dinamico, incrociando i dati della mappa con le segnalazioni recenti degli utenti per proporre mete suggerite in tempo reale, come l'avvistamento di una specie rara 
nelle vicinanze.

Parallelamente, per potenziare il lato \textit{Citizen Science} e ridurre le barriere all'ingresso per i non esperti, si ipotizza l'adozione di modelli di \textit{Machine Learning} 
direttamente sul dispositivo tramite TensorFlow Lite. Questa funzionalità consentirebbe il riconoscimento assistito della fauna inquadrata dalla fotocamera, validando istantaneamente 
le segnalazioni e migliorando la qualità dei dati raccolti.

Infine, l'esperienza potrebbe espandersi verso il sociale con l'implementazione di classifiche globali o tra amici e la 
condivisione delle proprie collezioni sui social network agirebbero da metodo promozionale per il Parco. Il culmine di questa evoluzione tecnologica sarebbe l'integrazione della Realtà 
Aumentata: inquadrando il paesaggio, l'utente vedrebbe sovrapposti i dati invisibili dei sensori IoT o le sagome virtuali della fauna locale, consolidando definitivamente il legame tra 
l'ambiente fisico e il suo gemello digitale.

A livello personale, il completamento di questo progetto rappresenta un traguardo di grande soddisfazione. Vedere un'idea iniziale trasformarsi in uno 
strumento digitale concreto, funzionante e in grado di unire l'innovazione tecnologica alla tutela ambientale, ha ripagato ampiamente l'impegno speso 
in tutte le fasi di sviluppo. EcoSpot è stata una sfida complessa ma molto stimolante, e aver sviluppato un prototipo pienamente funzionante, capace di 
dimostrare come la tecnologia possa avvicinare le persone alla natura, mi rende orgoglioso del risultato.