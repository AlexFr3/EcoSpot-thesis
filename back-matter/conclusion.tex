\chapter*{Conclusioni}
\addcontentsline{toc}{chapter}{Conclusioni}
\markboth{Conclusioni}{Conclusioni}

Il presente lavoro di tesi ha portato alla progettazione e allo sviluppo di \textit{EcoSpot}, un’applicazione mobile cross-platform realizzata in Flutter che si propone come uno 
strumento innovativo per la valorizzazione del Parco del Delta del Po. L'obiettivo primario è stato raggiunto integrando tecnologie eterogenee dalla cartografia digitale ai 
sensori IoT, fino ai servizi Cloud in un'unica interfaccia accessibile, capace di promuovere attivamente il turismo sostenibile e la consapevolezza ecologica.

L'analisi dei risultati dimostra come l'adozione di un'architettura a microservizi, basata su Riverpod per la gestione dello stato e su Firebase come backend, abbia garantito la 
scalabilità e la robustezza necessarie per gestire flussi di dati in tempo reale. In particolare, l'implementazione del paradigma del \textit{Digital Twin}, che unisce i dati 
ambientali (come livelli idrometrici e salinità) all'esperienza utente, ha permesso di rendere visibili le dinamiche spesso impercettibili dell'ecosistema lagunare. 
Questo approccio ha trasformato il visitatore da semplice osservatore a soggetto attivo nel monitoraggio del territorio tramite la \textit{Citizen Science} \cite{2025discover}.

Dal punto di vista della \textit{User Experience}, l'integrazione di un sistema di \textit{Gamification} fondato sulla \textit{Self-Determination Theory} \cite{ryan_deci_sdt} ha 
gettato le basi per un coinvolgimento profondo, incentivando l'esplorazione fisica dei luoghi tramite meccaniche di scoperta \textit{location-based}. Tuttavia, EcoSpot è un progetto 
in continua evoluzione; l'architettura modulare adottata permette di delineare diverse direttrici per sviluppi futuri volti ad arricchire ulteriormente l'esperienza immersiva nel Parco.

\section*{Sviluppi Futuri}

In ottica evolutiva, le funzionalità attuali possono essere estese introducendo nuove dinamiche di partecipazione per massimizzare la fidelizzazione dell'utente e l'impatto educativo 
del sistema. Mentre il prototipo attuale gratifica la scoperta singola con l'avanzamento di rango, un'evoluzione naturale prevede l'introduzione di missioni giornaliere e stagionali. 
Queste sfide, come la richiesta di documentare specifiche specie in un arco temporale definito o visitare determinate aree verdi, potrebbero essere accompagnate da un meccanismo di 
medaglie di completamento. L'obiettivo è premiare il collezionismo integrale della biodiversità censita, trasformando la visita in una sfida coinvolgente per completare il proprio 
archivio digitale e favorendo un coinvolgimento costante nel lungo periodo.

Per migliorare la fruibilità del territorio, si prevede inoltre l'integrazione di un sistema di \textit{routing} intelligente. A differenza della libera esplorazione, l'app potrebbe 
generare itinerari personalizzati basati sugli interessi specifici dell'utente o sulle sue capacità motorie, suggerendo percorsi accessibili e ottimizzati. Tale algoritmo potrebbe 
persino diventare dinamico, incrociando i dati della mappa con le segnalazioni recenti degli utenti per proporre mete suggerite in tempo reale, come l'avvistamento di una specie rara 
nelle vicinanze.

Parallelamente, per potenziare il lato \textit{Citizen Science} e ridurre le barriere all'ingresso per i non esperti, si ipotizza l'adozione di modelli di \textit{Machine Learning} 
direttamente sul dispositivo tramite TensorFlow Lite. Questa funzionalità consentirebbe il riconoscimento assistito della fauna inquadrata dalla fotocamera, validando istantaneamente 
le segnalazioni e migliorando la qualità dei dati raccolti.

Infine, l'esperienza potrebbe espandersi verso il socialecon l'implementazione di classifiche globali o tra amici e la 
condivisione delle proprie collezioni sui social network agirebbero da metodo promozionale per il Parco. Il culmine di questa evoluzione tecnologica sarebbe l'integrazione della Realtà 
Aumentata: inquadrando il paesaggio, l'utente vedrebbe sovrapposti i dati invisibili dei sensori IoT o sagome virtuali della fauna locale, consolidando definitivamente il legame tra 
l'ambiente fisico e il suo gemello digitale.