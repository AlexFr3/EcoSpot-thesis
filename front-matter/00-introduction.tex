\unchapter{Introduzione}
\addcontentsline{toc}{chapter}{Introduzione}
\markboth{Introduzione}{Introduzione}

Il Parco del Delta del Po costituisce uno degli ecosistemi più complessi e delicati del panorama europeo, 
rappresentando una Riserva di Biosfera MAB UNESCO di inestimabile valore naturalistico. Nonostante la 
straordinaria biodiversità e la varietà di habitat lagunari lo rendano una meta d'eccellenza per il 
turismo sostenibile, la fragilità di un territorio minacciato da fenomeni critici come l'intrusione 
del cuneo salino richiede soluzioni innovative per coinvolgere i visitatori nelle mutazioni ambientali 
in atto. Da questa urgenza nasce \textit{EcoSpot}, un'applicazione mobile sviluppata all'interno del 
progetto DISCOV.ER e concepita per fungere da ponte tra la dimensione tecnologica e quella naturalistica. 
L'obiettivo della tesi è proporre un sistema capace di trasformare la visita in un'attività dinamica e 
consapevole, integrando dati in tempo reale provenienti da sensori IoT per rendere visibili variabili 
ecologiche altrimenti impercettibili all'occhio umano.

Il cuore dell'applicativo risiede nella sinergia tra la raccolta partecipata dei dati e il coinvolgimento 
ludico dell'utente. Attraverso il contributo attivo dei cittadini nella generazione di segnalazioni 
georeferenziate, il visitatore cessa di essere uno spettatore passivo per diventare un custode della 
biodiversità. Parallelamente, l'adozione di meccaniche di gratificazione e sfide interattive incentiva 
l'esplorazione rispettosa del territorio, premiando l'acquisizione di competenze naturalistiche e 
rafforzando il legame con la comunità di tutela del Delta. EcoSpot non si limita dunque alla funzione 
di guida turistica, ma si configura come una piattaforma educativa avanzata che sfrutta il paradigma 
del \textit{Digital Twin} per connettere la realtà fisica del Parco alla sua rappresentazione digitale.

Il presente lavoro documenta l'intero processo di realizzazione, seguendo un percorso logico che 
parte dall'analisi del contesto per giungere alla soluzione ingegneristica finale.

L'analisi si apre con il primo capitolo, inquadrando lo scenario territoriale e scientifico di 
riferimento. Attraverso l'esame delle
peculiarità idrogeologiche e faunistiche dell'area, con particolare attenzione all'avifauna e alle 
problematiche idriche, si osserva l'evoluzione dell'offerta verso modelli di fruizione intelligente 
del patrimonio. In questo quadro vengono approfonditi i paradigmi abilitanti quali il monitoraggio 
collaborativo e l'uso di elementi motivazionali per incentivare comportamenti virtuosi. Tale studio 
preliminare si conclude posizionando l'app come una soluzione 
ibrida tra supporto informativo e risorsa scientifica.

Sulla base di queste premesse, si sviluppa il secondo capitolo sulla fase di progettazione e sull'analisi 
dei requisiti. In questa sezione vengono motivate le scelte architetturali che hanno portato all'adozione 
del framework Flutter per lo sviluppo multipiattaforma, garantendo un'esperienza fluida su 
diversi sistemi operativi. Parallelamente, viene descritta l'infrastruttura di backend affidata 
alla suite Google Firebase, selezionata per la sua capacità di gestire autenticazione e 
dati in tempo reale in modo scalabile.

Infine, nell'ultimo capitolo l'esposizione illustra nel dettaglio l'implementazione tecnica, 
approfondendo l'architettura 
reattiva gestita tramite Riverpod, essenziale per assicurare una gestione dello stato robusta. 
Vengono inoltre esposte le logiche spaziali impiegate per la cartografia interattiva e l'integrazione 
dei flussi IoT, mostrando come il codice traduca i presupposti teorici in funzionalità concrete. Lo 
studio si conclude con una valutazione critica dei risultati e una riflessione sugli sviluppi futuri, 
delineando come l'intelligenza artificiale e la realtà aumentata potrebbero ulteriormente potenziare 
l'impatto di EcoSpot sulla salvaguardia del territorio.