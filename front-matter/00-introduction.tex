\unchapter{Introduzione}
\addcontentsline{toc}{chapter}{Introduzione}
\markboth{Introduzione}{Introduzione}

Il Parco del Delta del Po costituisce uno degli ecosistemi più complessi e delicati del panorama europeo, 
rappresentando una Riserva di Biosfera MAB UNESCO di inestimabile valore naturalistico. 
Focalizzando l'attenzione in particolare sul versante emiliano-romagnolo, il Parco offre un 
mosaico paesaggistico unico che spazia dalle vaste zone umide delle Valli di Comacchio e dalle antiche 
Saline di Cervia, fino alle fitte foreste del Bosco della Mesola e alle storiche pinete costiere. 
Questa straordinaria varietà di habitat lagunari, fluviali e boschivi lo rende una meta d'eccellenza per il turismo sostenibile e un rifugio 
vitale per innumerevoli specie animali. L'avifauna rappresenta senza dubbio una delle ricchezze più celebri del territorio, che ospita eleganti 
colonie di fenicotteri rosa, oltre a specie caratteristiche e protette come l'avocetta, il cavaliere d'Italia, 
il cormorano e il falco di palude.

Tuttavia, la fragilità di questo ecosistema e la sopravvivenza dei volatili sono oggi messe a dura prova dai mutamenti climatici e dalle alterazioni idrogeologiche. 
Tra i fenomeni più critici e insidiosi spicca l'intrusione del cuneo salino, ovvero la progressiva risalita delle acque marine all'interno degli alvei fluviali del delta e 
nelle falde acquifere sotterranee. Tale dinamica, innescata e aggravata dai prolungati periodi di siccità, dalla drastica riduzione della portata d'acqua dolce dei fiumi 
e dal fenomeno della subsidenza, altera profondamente il delicato equilibrio tra acqua dolce e salmastra. Questo sbilanciamento degrada rapidamente le aree di 
nidificazione e di foraggiamento, minacciando direttamente la fauna e la flora, oltre a rischiare di compromettere irrimediabilmente le attività agricole del territorio limitrofo.

In questo scenario di vulnerabilità, dove la tutela della biodiversità aviaria diventa un'emergenza prioritaria, l'osservazione passiva 
non è più sufficiente; si rende necessario un monitoraggio scientifico costante e capillare. In questa cornice si inserisce il progetto DISCOV.ER, 
un'iniziativa avanzata di ricerca e innovazione tecnologica mirata alla creazione di un Digital Twin dell'ecosistema del Delta del Po. 
Il progetto si avvale di una vasta rete di sensori \ac{IoT} immersi direttamente nei corpi idrici e dislocati in punti strategici. Questi dispositivi sono programmati per rilevare e 
trasmettere in tempo reale dati ambientali, tra cui i livelli di salinità dell'acqua, le variazioni di temperatura, la conducibilità, 
la velocità del vento e ulteriori parametri di rilievo.

La disponibilità di questa mole di dati richiede però strumenti digitali adeguati affinché possa essere resa 
facilmente accessibile e consultabile anche dal grande pubblico. 
Dati invisibili come quelli sul cuneo salino nascondono infatti impatti devastanti sulla sopravvivenza degli animali. 
È proprio da questa necessità di 
tutelare il patrimonio faunistico e dare voce all'ecosistema che nasce \textit{EcoSpot}, un'applicazione 
mobile sviluppata all'interno del progetto DISCOV.ER e concepita per fungere da ponte tra la complessa 
dimensione tecnologica di monitoraggio e l'esperienza naturalistica del visitatore. L'app offre la possibilità di visionare 
le metriche dei sensori \ac{IoT} sulla mappa in modo chiaro e di immediata comprensione. 
Rendendo tangibile l'impatto dei cambiamenti ambientali sugli habitat e sulle specie che li popolano, l'applicazione 
informa l'utente e lo rende pienamente consapevole dell'ecosistema che sta attraversando.

La parte centrale di EcoSpot, tuttavia, risiede nell'obiettivo finale di questa consapevolezza: la salvaguardia attiva della fauna. 
Per incentivare questo principio l'applicazione integra meccaniche di gamification e sfide interattive studiate per incentivare 
un'esplorazione responsabile e rispettosa del territorio.
Attraverso il completamento delle sfide proposte, il visitatore viene attivamente coinvolto, sviluppando una maggiore propensione al rispetto e alla tutela dell'ecosistema.
EcoSpot non si limita dunque alla funzione di guida turistica, ma si configura come una piattaforma educativa avanzata che sfrutta l'infrastruttura di DISCOV.ER 
per connettere in modo bidirezionale la realtà fisica del Parco alla sua rappresentazione digitale.

Il presente lavoro documenta l'intero processo di realizzazione dell'applicazione, seguendo un percorso logico che parte dall'analisi del contesto 
per giungere alla soluzione ingegneristica finale.
L'analisi si apre con il primo capitolo, inquadrando lo scenario territoriale di riferimento. 
Attraverso l'esame delle peculiarità idrogeologiche e faunistiche dell'area, con un focus centrale sulla tutela dell'avifauna, si osserva l'evoluzione 
dell'offerta turistica verso modelli di fruizione intelligente del patrimonio. 
In questo quadro vengono approfonditi i paradigmi abilitanti quali il monitoraggio collaborativo e l'uso di elementi motivazionali per 
incentivare comportamenti responsabili. Tale studio preliminare si conclude posizionando l'app come uno strumento innovativo per 
la divulgazione e il coinvolgimento attivo del visitatore.

Sulla base di queste premesse, si sviluppa il secondo capitolo, che si apre con l'analisi dei requisiti funzionali e non funzionali, 
per poi passare alla vera e propria fase di progettazione. 
A fronte delle necessità emerse, vengono motivate le scelte architetturali adottate: da un lato l'utilizzo del framework Flutter per uno 
sviluppo multipiattaforma capace di garantire un'esperienza fluida su vari sistemi operativi; dall'altro la scelta della suite Google Firebase 
come infrastruttura di backend, ideale per gestire autenticazione e dati in tempo reale con un'elevata scalabilità.

Infine, nell'ultimo capitolo la tesi illustra nel dettaglio l'implementazione tecnica dell'applicativo, approfondendo 
l'architettura gestita tramite Riverpod, essenziale per assicurare una gestione dello stato dell'interfaccia solida e manutenibile. 
Vengono inoltre esposte le logiche impiegate per la mappa interattiva e l'integrazione a livello di codice dei flussi di dati provenienti dai sensori IoT, 
mostrando in che modo siano stati tradotti in funzionalità concrete. 

Lo studio si conclude con le considerazioni finali sul lavoro svolto e una riflessione sugli sviluppi futuri, 
ipotizzando come l'integrazione di tecnologie quali l'intelligenza artificiale e la realtà aumentata possa ulteriormente potenziare 
l'impatto di EcoSpot sulla salvaguardia del patrimonio faunistico.