% !TEX root = ../thesis-main.tex
\chapter{Inquadramento territoriale e scenario di riferimento}
\label{chap:context}

\section{La Riserva di Biosfera MAB UNESCO: un ecosistema da valorizzare}
Il Parco del Delta del Po costituisce la più vasta zona umida d’Italia e una delle aree europee a più alta biodiversità, grazie al mosaico di lagune salmastre, valli da pesca, canneti, pinete costiere, dune e zone agricole. Questa varietà di habitat sostiene un numero elevatissimo di specie vegetali e animali, motivo per cui l’area è riconosciuta come Riserva di Biosfera MAB UNESCO \cite{unesco_po_official} \cite{biosfera_mab_reason}. Dal punto di vista faunistico, il Delta ospita oltre 300–350 specie di uccelli tra nidificanti, svernanti e migratori, rendendolo uno dei siti più importanti d’Europa per l’avifauna acquatica. Tra le specie più caratteristiche figurano i fenicotteri rosa (Phoenicopterus roseus), divenuti simbolo del paesaggio lagunare, insieme ad aironi cenerini, cormorani, cavalieri d’Italia, avocette, falchi di palude e molte altre specie legate alle zone umide. I fenicotteri sfruttano le acque basse e salmastre delle valli e delle lagune per alimentarsi, formando spesso grandi concentrazioni di individui che possono raggiungere diverse migliaia di esemplari nei periodi di maggior presenza. Questa specie è considerata particolarmente protetta a livello internazionale ed è indicativa di buone condizioni ecologiche degli ambienti umidi costieri. Accanto all’avifauna, il Parco ospita una ricca fauna ittica (circa 60 specie, alcune tipiche del Delta), numerosi invertebrati legati alle acque e ai sedimenti, e una flora dominata da canneti, flora adattata agli ambienti salmastri e ultimi lembi delle antiche foreste di pianura , che contribuiscono a funzioni ecosistemiche chiave come depurazione delle acque, protezione costiera e stoccaggio di carbonio. L’insieme di questi elementi fa del Delta del Po un laboratorio naturale per lo studio e la gestione della biodiversità lagunare in un contesto in cui conservazione e attività umane (agricoltura, pesca, turismo) devono coesistere in modo sostenibile.

\subsection{Biodiversità e patrimonio naturalistico lagunare}
L'area è un habitat cruciale per numerose specie, tra cui il Fenicottero Rosa (\textit{Phoenicopterus roseus}), simbolo della biodiversità locale. La flora comprende specie alofile come la Salicornia, adattate agli ambienti salmastri tipici delle lagune costiere.

\subsection{Specificità idro-morfologiche del versante emiliano-romagnolo}

\paragraph{Il relitto forestale della Mesola}
Riserva naturale orientata, rappresenta l'ultimo residuo delle antiche foreste termofile litoranee che un tempo ricoprivano la costa adriatica.

\paragraph{Il sistema delle Valli di Comacchio}
Un complesso di lagune salmastre dove la gestione idraulica è fondamentale. È qui che il progetto EcoTwin concentra parte del monitoraggio, in particolare nelle stazioni di Bellocchio e Foce, aree strategiche per l'avifauna \cite{parcodelta}.

\paragraph{La pressione antropica sulla fascia costiera}
Aree soggette a forte pressione antropica, che necessitano di strumenti innovativi per veicolare i flussi turistici verso forme di fruizione più sostenibili e consapevoli.

\subsection{Evoluzione della domanda turistica verso la fruizione consapevole}
I visitatori moderni richiedono informazioni puntuali e in tempo reale. Nell'ottica dello \textit{Smart Tourism} \cite{smart_tourism}, non è sufficiente indicare i punti di interesse statici; è necessario fornire dati contestuali sulle condizioni ambientali (es. livelli dell'acqua, meteo specifico) per pianificare l'escursione in sicurezza.

\section{Paradigmi tecnologici per la gestione delle aree protette}

\subsection{Sistemi di geolocalizzazione e mappatura dinamica}
Le mappe digitali hanno progressivamente sostituito quelle cartacee, permettendo la geolocalizzazione dinamica dell'utente e l'interazione con i Punti di Interesse (POI) distribuiti sul territorio.

\subsection{Il monitoraggio partecipativo attraverso la Citizen Science}
Il coinvolgimento dei cittadini nella raccolta dati, noto come \textit{Citizen Science} \cite{citizen_science}, trasforma il turista da osservatore passivo a partecipante attivo nel monitoraggio della biodiversità, contribuendo alla ricerca scientifica.

\subsection{Meccaniche ludiche per l'incentivazione territoriale}
La \textit{Gamification} è definita come l'uso di elementi di game design in contesti non ludici \cite{gamification_def}. Nel progetto EcoTwin, meccaniche quali punti esperienza (XP), livelli e badge (es. "Leggenda del Delta") vengono impiegate per incentivare l'esplorazione fisica di nuovi habitat.

\section{Caso di studio: l'ecosistema digitale EcoTwin}

\subsection{La sintesi tra dati ambientali ed esperienza utente}
L'obiettivo principale è la creazione di un "Digital Twin" \cite{digital_twin_env} informativo che colleghi i dati oggettivi dei sensori idrografici (livello canali, salinità) all'esperienza soggettiva del visitatore.

\subsection{Architettura del sistema e integrazione IoT}
L'architettura proposta si basa su un'applicazione mobile sviluppata in Flutter \cite{flutter}, collegata a un backend Cloud per la sincronizzazione dei progressi e a un'infrastruttura IoT per l'acquisizione dei dati ambientali in tempo reale.

\subsection{Impatto atteso sull'engagement e sulla consapevolezza ecologica}
Si attende un aumento dell'engagement dei visitatori e una maggiore consapevolezza ecologica, dimostrando come le tecnologie digitali possano supportare efficacemente la gestione e la fruizione delle aree protette.